\documentclass[twocolumn]{article}
\usepackage[left=10mm,right=11mm,top=0mm,bottom=20mm]{geometry}
\usepackage{fourier}
\usepackage[T1]{fontenc}
\RequirePackage[scaled=0.88]{luximono}
\usepackage{listings}

\lstset{language=Caml,
  columns=flexible,
  basicstyle={\rmfamily},
  morekeywords={rule,parse,eof,val}
}

\frenchspacing

\title{COMS W4115 \\
Programming Languages and Translators \\
Homework Assignment 1}
\author{
\begin{tabular}{ll}
Prof. Stephen A. Edwards &  Due October 1st, 2018 \\
Columbia University & at 4:00 PM \\
\end{tabular}
}
\date{}

\pagestyle{empty}

\begin{document}
\maketitle

Submit your assignment as a single PDF file on Courseworks.
\textbf{Include a demonstration of your code working on examples},
e.g., by including a screenshot of your code compiling and working.

Do this assignment alone.  You may consult the instructor or a TA, but
not other students.  All the problems ask you to use OCaml.  You may
download the compiler from ocaml.org.

\begin{enumerate}

\item Write an OCaml function \texttt{maxrun} that reports the length
  of the longest contiguous run of equal values in a list.  E.g.,

\begin{lstlisting}
val maxrun : 'a list -> int = <fun>
# maxrun [];;
- : int = 0
# maxrun [1];; 
- : int = 1
# maxrun [1;1];;
- : int = 2
# maxrun [1;1;2;2;2;1;3;3];;
- : int = 3
\end{lstlisting}

% let all f l = List.fold_left (&&) true (List.map f l)

% let rec all f = function
%   [] -> true
% | h::t -> f h && all f t

\vspace{-0.5\baselineskip}

\item Write a word frequency counter.  Start from the following
  ocamllex program (wordcount.mll) that gathers in a list of strings
  all the words in a file, then prints them.

\begin{lstlisting}
{ type token = EOF | Word of string }

rule token = parse
  | eof { EOF }
  | ['a'-'z' 'A'-'Z']+ as word { Word(word) }
  | _ { token lexbuf }

{
 let lexbuf = Lexing.from_channel stdin in
 let wordlist = 
   let rec next l =
     match token lexbuf with
         EOF -> l
     | Word(s) -> next (s :: l)
   in next []
 in
 List.iter print_endline wordlist
}
\end{lstlisting}

Replace the List.iter call with code that scans through the list and
builds a string map whose keys are words and whose values count the
number of apearances of each word.  Then, use StringMap.fold to
convert this to a list of (count, word) tuples; sort them using
List.sort; and print them with List.iter.  Sort the list of (count,
word) pairs using

\begin{lstlisting}
 let wordcounts =
   List.sort (fun (c1, _) (c2, _) ->
              Pervasives.compare c2 c1)
     wordcounts in
\end{lstlisting}

Compiling and running my (20-more-line) solution:

\begin{verbatim}
$ ocamllex wordcount.mll
4 states, 315 transitions, table size 1284 bytes

$ ocamlc -o wordcount wordcount.ml

$ ./wordcount < wordcount.mll

9 word
7 map
7 let
7 StringMap
6 in
...
\end{verbatim}

\item Extend the three-slide ``calculator'' example shown in the OCaml
  slides (the source is also available on the class website) to accept
  variables named with identifiers consisting of lowercase letters,
  assignment to those variables, and sequencing using the ``;''
  operator.  For example,

\begin{verbatim}
foo = 3; bar = baz = 6; foo * bar + baz
\end{verbatim}

should print ``24''

Use a string-to-integer Map to track variable variables.  Add tokens
to the parser and scanner for representing assignment, sequencing, and
variable names.

The ocamllex rule for the variable names, which converts the letters a--z
into the corresponding literals, is

\begin{lstlisting}
| ['a'-'z']+ as id { VARIABLE(id) }
\end{lstlisting}

The new ast.mli file is

\begin{lstlisting}
type operator = Add | Sub | Mul | Div
type expr =
    Binop of expr * operator * expr
  | Lit of int
  | Seq of expr * expr
  | Asn of string * expr
  | Var of string
\end{lstlisting}

\textbf{Make sure your code compiles without warnings}

\end{enumerate}

\thispagestyle{empty}
\end{document}

% Local Variables:
% compile-command: "make hw1.pdf"
% End:
